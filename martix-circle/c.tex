\def\mytitle{MATRICES USING PYTHON}
\def\myauthor{T.MANASAREDDY}
\def\contact{manasatanuboddi@gmail.com}
\def\mymodule{Future Wireless Communication (FWC)}
\documentclass[10pt, a4paper]{article}
\usepackage[a4paper,outer=1.5cm,inner=1.5cm,top=1.75cm,bottom=1.5cm]{geometry}
\twocolumn
\usepackage{graphicx}
\graphicspath{{./images/}}
\usepackage[colorlinks,linkcolor={black},citecolor={blue!80!black},urlcolor={blue!80!black}]{hyperref}
\usepackage[parfill]{parskip}
\usepackage{lmodern}
\usepackage{tikz}
	\usepackage{physics}
%\documentclass[tikz, border=2mm]{standalone}
%\usepackage{karnaugh-map}
%\documentclass{article}
\usepackage{tabularx}
%\usepackage{circuitikz}
\usepackage{enumitem}
\usetikzlibrary{calc}
\usepackage{amsmath}
\usepackage{amssymb}
\renewcommand*\familydefault{\sfdefault}
\usepackage{watermark}
\usepackage{lipsum}
\usepackage{xcolor}
\usepackage{listings}
\usepackage{float}
\usepackage{titlesec}
\providecommand{\mtx}[1]{\mathbf{#1}}
\titlespacing{\subsection}{1pt}{\parskip}{3pt}
\titlespacing{\subsubsection}{0pt}{\parskip}{-\parskip}
\titlespacing{\paragraph}{0pt}{\parskip}{\parskip}
\newcommand{\figuremacro}[5]{
    \begin{figure}[#1]
        \centering
        \includegraphics[width=#5\columnwidth]{#2}
        \caption[#3]{\textbf{#3}#4}
        \label{fig:#2}
    \end{figure}
}

\newcommand{\myvec}[1]{\ensuremath{\begin{pmatrix}#1\end{pmatrix}}}
\let\vec\mathbf
\lstset{
frame=single, 
breaklines=true,
columns=fullflexible
}
\thiswatermark{\centering \put(181,-119.0){\includegraphics[scale=0.13]{iith_logo3}} }
\title{\mytitle}
\author{\myauthor\hspace{1em}\\\contact\\FWC22048\hspace{6.5em}IITH\hspace{0.5em}\mymodule\hspace{6em}Assignment}
\begin{document}
	\maketitle
	\tableofcontents
   \section{Problem}
  If the lines 3x-4y-7=0 and 2x-3y-5=0 are two diameters of a circle of area 49 $\pi$ square units,the equation of the circle is 
\section{Construction}
  \includegraphics[scale=0.47]{figure_1.png}
  	\begin{center}
  Figure of construction
  	\end{center}
  \section{Solution}
The area of the Circle is 49 $\pi$\\
Let r be the radius of circle,\\
\begin{equation}
\pi r^2 = 49\pi 
\end{equation}
\begin{equation}
r=7
\end{equation}
The diameter equations are:\\
\begin{align}
3x-4y-7=0
\end{align}
\begin{align}
	2x-3y-5=0
	\end{align}
	From the above we obtain the matrix equations: \\
	\center
\myvec{ 3 & -4 \\ 2 & -3} \myvec{x \\ y} = \myvec{7\\5} \\
\center
\begin{flushleft}
The augmented matrix can be expressed as,
\end{flushleft}
\myvec{ 3 & -4 & 7 \\ 2 & -3 & 5} \\
The standard equation of the conics is given as :
\myvec{ 1 & 0 & 1 \\ 0 & 1 & -1} \vspace{0.3cm}
\begin{flushleft}
From this we can find the center points of x and y
\end{flushleft}
\myvec{ x \\ y} =\myvec{1 \\-1}\\
\begin{center}
$\vec{c}$= \myvec{1 \\-1} \\
\begin{flushleft}
The standard equation of the conics is given as :
\end{flushleft}
\begin{align}
\vec{x}^{\top}\vec{V}\vec{x}+2\vec{u}^{\top}\vec{x}+f=0
\end{align}
\center
  $\implies$  $ \vec{x}^T$$\vec{I}$ $\vec{x}$  + 2 $ \myvec{-1\\1}^T \vec{x} -47 = 0$
\endcenter
\begin{align}
	\vec{V} &= \vec{I}, \vec{u} = -\myvec{1 \\-1}, f = -47
	\end{align}
\subsection{Deriving equation for Circle in quadratic form}
\center
  $$  (x-x1)^2+(y-y1)^2 = r^2 $$\\
    $$(x-1)^2+(y+1)^2 = 7^2$$
\endcenter
\begin{equation}
    x^2+y^2-2x+2y-47=0
\end{equation}	
\begin{lstlisting}
bash bash2.sh............using shell command
\end{lstlisting}
\begin{center}
Below python code realizes the above construction :
\fbox{\parbox{8.5cm}{\url{https://github.com/manasareddy/FWC_module1/blob/main/matrices/circle/codes/matrix.py}}}
\end{center}
\end{document}
